\chapter{Completed Work}

\section{The technology of choice}

One of the most popular programming languages in web development is JavaScript. The usage of this frontend focused technology is growing quickly.  It is the 7th on Tiobe Index, which is a good indicator of programming languages popularity. \cite{tiobe}

Learning and teaching JavaScript, HTML and CSS is important. My tool focuses on this three main building blocks of the web.

Building a frontend heavy application, with a dynamic, user-friendly interface is more common nowadays. In the last few years JavaScript based frontend frameworks became a matured, production ready tools. Server side technologies, like database management and time and resource heavy processes are separated from the user focused, design driven view layer, which is developed with usage of frontend frameworks.

The most popular tools are Angular.js, React.js and Ember.js. In my project I use Ember.js. It is an "opinionated" framework. Opinionet, convention over configuration driven framework means, that developers should follow specific conventions, instead of freely using the tool. A more strict environment helps to adapt the best practices and speed up the development process also.

Certainly, we still have to store data and information, so we cannot live without backend and server technology. Luckily there are already cloud-based tools for managing databases. I use Firebase, which is a service provided by Google. Firebase is a cloud-based database, document-store solution and easy to integrate with Ember.js.

Additionally, I build a traditional backend server application also to support development and experimenting with a real server side and a cloud-based solutions parallel. My preferred technology on backend side is Ruby on Rails, a popular, also opinionated and convention over configuration driven backend framework.


\section{The development environment}

Following the most modern standard of web applications, I separate the user faced frontend development and the data store, backend development.

The user face frontend application uses Ember.js frontend framework. The primer data store is Firebase cloud based database, however I already created a Ruby on Rails application which is running when the application is in development mode.

Ember.js development requires Node.js on the development machine to run the development environment. This development environment helps to run and modify frontend code quickly and it generates the final, deployable production code also. The production version of the application is only a static website. It means, there is one index.html, two JavaScript files and two CSS files.

\section{Frontend features}

The look and feel of the application will follow the standard Bootstrap style. Bootstrap is added to the project. I uses SaSS version of the Bootstrap, so I can customize it with the modification of the SaSS variables. SaSS is a modern CSS development environment, helps to programmatically modify the CSS.

The home page of the application is only a placeholder. I added a navigation bar with the following links: Home, Sandbox, Tutorials, Admin.

I implemented a breadcrumb bar also, which helps in navigation.

I created first a Sandbox area, where I experiment with the CodeMirror code editor and an IFrame, which shows the preview. This Sandbox page contains the editor. When the source code is updated, the preview page automatically shows the generated website. This features uses Ember.js default two-way bindings capability.

Database management is already implemented. The main adapter is the Firebase adapter, which automatically update data to the Firebase server. Firebase is a real time database. The limited, free to use version is enough for experimenting and for demo.

The secondary adapter is a JSONApi Adapter. JSON Api \cite{jsonapi} is a new standard of data communication format. This is the Ember.js default adapter.

\section{Future work}
