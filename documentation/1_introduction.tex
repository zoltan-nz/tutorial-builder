\chapter{Introduction}

This project is about building an online publishing tool prototype, using code editors and step by step instructions to present programming challenges or solution for a computer science problem.

The prototype has two parts, an administration area, where the content creator can build a tutorial, and a player tool, where the recorded steps would be presented. This player should be embeddable in any blog post or website.

The primary target user is the creator, who composes a new tutorial. The creator can be a teacher, or an open source project owner, who would like to introduce his tool or code.

The secondary user is the consumer, who wants to learn or know more about a problem or a coding solution.

\section{Motivation}

We all have the unstoppable desire to learn. We are keen to know more about the world around us, about our hobby and our profession. In software development, in computer science, the knowledge is essential, it is the key to succeeding. Reading, studying, sharing. An infinite loop of collecting and adapting new practices.

In information technology, especially in programming languages, writing blog posts, creating static, step by step tutorials are a popular way to share or learn something new. Producing and sharing the content is easier nowadays, but still required more effort from the creator, when they want to deliver an easy to understand high-quality tutorials.

Creating interactive tutorials are appealing, but the production cost is much higher. Recording a video tutorial or especially updating it is time-consuming and involves more effort from the creator.

I think an ideal solution would be a healthy mix of static and dynamic contents. Where the learner can read instructions but meanwhile can watch the steps in a code editor, in a more realistic environment.
