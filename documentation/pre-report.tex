%% $RCSfile: proj_report_outline.tex,v $
%% $Revision: 1.3 $
%% $Date: 2016/06/10 03:41:54 $
%% $Author: kevin $

\documentclass[11pt
              , a4paper
              , twoside
              , openright
              ]{report}


\usepackage{float} % lets you have non-floating floats

\usepackage{url} % for typesetting urls

%
%  We don't want figures to float so we define
%
\newfloat{fig}{thp}{lof}[chapter]
\floatname{fig}{Figure}

%% These are standard LaTeX definitions for the document
%%
\title{Preliminary Report}
\author{Zoltan Debre}

%% This file can be used for creating a wide range of reports
%%  across various Schools
%%
%% Set up some things, mostly for the front page, for your specific document
%
% Current options are:
% [ecs|msor|sms]          Which school you are in.
%                         (msor option retained for reproducing old data)
% [bschonscomp|mcompsci]  Which degree you are doing
%                          You can also specify any other degree by name
%                          (see below)
% [font|image]            Use a font or an image for the VUW logo
%                          The font option will only work on ECS systems
%
\usepackage[image,ecs,mcompsci]{vuwproject}

% You should specifiy your supervisor here with
\supervisor{Dr Stuart Marshall}
% use \supervisors if there is more than one supervisor

% Unless you've used the bschonscomp or mcompsci
%  options above use
%   \otherdegree{OTHER DEGREE OR DIPLOMA NAME}
% here to specify degree

% Comment this out if you want the date printed.
\date{}

\begin{document}

% Make the page numbering roman, until after the contents, etc.
\frontmatter

%%%%%%%%%%%%%%%%%%%%%%%%%%%%%%%%%%%%%%%%%%%%%%%%%%%%%%%

%%%%%%%%%%%%%%%%%%%%%%%%%%%%%%%%%%%%%%%%%%%%%%%%%%%%%%%

\begin{abstract}

My project is about building an online publishing tool prototype, using code editors and step by step instructions to present programming challenges or solution for a computer science problem.

The prototype has two parts, an administration area, where the content creator can build a tutorial, and a player tool, where the recorded steps would be presented. This player should be embeddable in any blog post or website.

\end{abstract}

%%%%%%%%%%%%%%%%%%%%%%%%%%%%%%%%%%%%%%%%%%%%%%%%%%%%%%%

\maketitle

%\chapter*{Acknowledgments}\label{C:ack} 
Any acknowledgments should go 
in here, between the title page and the table of contents.  The 
acknowledgments do not form a proper chapter, and so don't get a 
number or appear in the table of contents.

\tableofcontents

% we want a list of the figures we defined
\listof{fig}{Figures}

%%%%%%%%%%%%%%%%%%%%%%%%%%%%%%%%%%%%%%%%%%%%%%%%%%%%%%%

\mainmatter

%%%%%%%%%%%%%%%%%%%%%%%%%%%%%%%%%%%%%%%%%%%%%%%%%%%%%%%

% individual chapters included here
\chapter{Introduction}\label{C:intro}
%This chapter gives an introduction to the project report.

%In Chapter \ref{C:us} we explain how to use this document, and the \texttt{vuwproject} style. In Chapter \ref{C:ex} we say some things about \LaTeX, and in Chapter \ref{C:con} we give our conclusions.

My project is about building an online publishing tool prototype, using code editors and step by step instructions to present programming challenges or solution for a computer science problem.

The prototype has two parts, an administration area, where the content creator can build a tutorial, and a player tool, where the recorded steps would be presented. This player should be embeddable in any blog post or website.

My project primary target user is the creator, who composes a new tutorial. The creator can be a teacher, or an open source project owner, who would like to introduce his tool or code.

The secondary user is the consumer, who want to learn or know more about a problem or a coding solution.

\section{Background}

We all have the unstoppable desire to learn. We are keen to know more about the world around us, about our hobby and our profession. In software development, in computer science, the knowledge is essential, it is the key to succeeding. Reading, studying, sharing. An infinite loop of collecting and adapting new practices.

In information technology, especially in programming languages, writing blog posts, creating static, step by step tutorials are a popular way to share or learn something new. Producing and sharing the content is easier nowadays, but still required more effort from the creator, when they want to deliver easy to understand, high quality tutorials.

Creating interactive tutorials are appealing, but the production cost is much higher. Recording a video tutorial or especially updating it is time-consuming and involves more effort from the creator.

I think an ideal solution would be a healthy mix of static and dynamic contents. Where the learner can read instructions but meanwhile can watch the steps in a code editor, in a more realistic environment.

%% $RCSfile: using.tex,v $
%% $Revision: 1.1 $
%% $Date: 2010/04/23 01:57:05 $
%% $Author: kevin $
%%
\chapter{Using this document and the \texttt{vuwproject} style}\label{C:us}

If you are writing an MSc or PhD thesis you should \emph{not} be using this style. Instead use \verb=vuwthesis=, which is based on the book style, and conforms to the VUW thesis rules. The thesis style is rather different from the project report style. 

This document is formatted using a local (to ECS and MSOR at VUW) style file. When you write your project report you should be very careful when changing the beginning. The document class settings should read:

\begin{verbatim}
\documentclass[11pt
              , a4paper
              , twoside
              , openright
              ]{report}
\end{verbatim}
The options to the document class specify that:
\begin{itemize}
\item 11pt font is to be used for the main body text,
\item  we will print on A4 paper, 
\item we will use duplex (two-sided) printing,
\item we want chapters to start on a right-hand page. 
\end{itemize}

The opitons you supply to the  \texttt{vuwproject} style will depend upon
what you are using the style for.

\subsection{Specifying the details}
The \texttt{vuwproject} style sets up the front page properly, and provides various commands allowing you to specify the author, title, supervisor or supervisors, the school from which the report is being submitted and the degree that the report is being submitted for. The style has deliberately been designed to do as little as possible. This means that your document can easily be re-formatted as a technical report, or for submission to a conference or journal by using the appropriate style.

It is also possible to use the style to easily produce documents on a
stand-alone computer where your \LaTeX installtion might not have all
of the  files and fonts available to machines within ECS or MSOR.

Most of the options to the \texttt{vuwproject} style are currently a simple
choice and there's a default that will make it obvious if you do not make
a choice.

Use one of the following options to use fonts available on ECS/MSOR machines
or to use images that imitate them (assumes you have copies of the images)
\begin{itemize}
\item \verb+font+
\item \verb+image+
\end{itemize}

Use one of the following options to set the school,
\begin{itemize}
\item \verb+ecs+
\item \verb+msor+
\end{itemize}

Use one of the following options to choose a pre-defined degree,
\begin{itemize}
\item \verb+bschonscomp+
\item \verb+mcompsci+
\end{itemize}

or use this command to use an explicit degree or diploma name
\begin{itemize}
\item \verb+\otherdegree{DEGREE OR DIPLOMA NAME}+
\end{itemize}

So, for example, to submit a report for the Master of Comp Sci degree, which
the style knows about, from within ECS, using the images, you'ld ensure the
 \texttt{vuwproject} line options looked like:

\begin{verbatim}
\usepackage[image,ecs,mcompsci]{vuwproject}
\end{verbatim}

whereas for a degree from within MSOR, when creating the final version on
an ECS or MSOR machine where you have access to the fonts, you would use
these options

\begin{verbatim}
\usepackage[font,msor]{vuwproject}
\end{verbatim}


and add the other degree's name using this command 

\begin{verbatim}
\otherdegree{DEGREE OR DIPLOMA NAME}
\end{verbatim}

To specify the supervisor or supervisors use either of the following commands in the preamble.
\begin{itemize}
\item \verb+\supervisor{The Supervisor}+
\item \verb+\supervisors{Super 1 and Super 2}+
\end{itemize}

If you fail to set any degree or supervisor, or the school, then the front page will report this.

The \texttt{vuwproject} style also sets the default font to be Palatino, using the \texttt{mathpazo} package. Palatino is one of VUW's `offical' fonts, and is the font used for the heading on the front page. The \texttt{mathpazo} package also typesets maths in a style which suits Palatino. 

\section{Copying the style}
If you want to write your project report away from VUW you will need to make your own copy of the \texttt{vuwproject} style.

You can find out where the original lives by reading the messages that \LaTeX\ prints when it is run.

Alternatively, you can down load a copy of the  \texttt{vuwproject} style from
the ECS webpages.

Any changes made to your own copy of the \texttt{vuwproject} style will not be reflected in the original, and \textit{vice versa}. Hence it makes sense to leave this as it is, and use a local style file for your own definitions.   

\chapter{Some \LaTeX\ hints and tips}\label{C:ex}
\LaTeX\ is a very good tool for producing well-structured documents 
carefully. It is very bad tool for banging things together in a rush 
and panic. 

\section{Floats}
One perennial problem with \LaTeX\ is its treatment of 
\emph{floats}.  Suppose you have a figure or table which you want to 
include in your document. Where should it go? Traditional typesetting 
practice is to put these in some convenient place, such as the top or 
bottom of the current or next page, or at the end of the section or 
chapter.  \LaTeX\ adopts a similar strategy, and allows floats to 
``float'' away from where they were defined. You can give a hint 
about where you want the figure, but \LaTeX\ may move it. Sometimes 
this is fine but sometimes you may want to have more control and 
insist that a float goes \emph{here}. Anselm Lingau's 
\textsf{float} package gives you this flexibility. For example, the following figure is an example of a non-floating float:

\begin{fig}[H]
\begin{center}
\begin{tabular}{l|lll}
$\delta$ & $\mathit{a}$ & $\mathit{b}$ & $\Lambda$ \\ \hline 
$S_{1}$  & $\{\}$       & $\{\}$      & $\{S_{2}, S_{5}, S_{10}\}$\\
$S_{2}$  & $\{S_{3}\}$  & $\{\}$      & $\{\}$\\
$S_{3}$  & $\{S_{4}\}$  & $\{\}$      & $\{\}$\\
$S_{4}$  & $\{S_{3}\}$  & $\{\}$      & $\{\}$\\
$S_{5}$  & $\{\}$       & $\{S_{6}\}$ & $\{\}$\\
$S_{6}$  & $\{\}$       & $\{S_{7}\}$ & $\{S_{8}\}$\\
$S_{7}$  & $\{S_{6}\}$  & $\{\}$      & $\{\}$\\
$S_{8}$  & $\{S_{9}\}$  & $\{\}$      & $\{\}$\\
$S_{9}$  & $\{\}$       & $\{S_{8}\}$ & $\{\}$\\
$S_{10}$ & $\{S_{11}\}$ & $\{\}$      & $\{\}$\\
$S_{11}$ & $\{\}$       & $\{S_{10}\}$& $\{\}$\\ 
\end{tabular}
\caption{The transition function of an NFA with $\Lambda$  transitions}

\end{center}
\end{fig}

On the other hand, Figure \ref{Fig:two} is a floating float. 



\begin{fig}[tbh]
\begin{center}
\begin{tabular}{l|ll}
$\delta''$ & $\mathit{a}$ & $\mathit{b}$ \\ \hline 
$T_{1}$  & $T_{2}$ & $T_{3}$\\ 
$T_{2}$  & $T_{4}$ & $T_{5}$\\ 
$T_{3}$  & $T_{6}$ & $T_{7}$\\ 
$T_{4}$  & $T_{8}$ & \\
$T_{5}$  & $T_{10}$ & \\
$T_{6}$  &  & $T_{11}$\\ 
$T_{7}$  & $T_{3}$ & \\
$T_{8}$  & $T_{4}$ & \\
$T_{10}$  &  & $T_{5}$\\ 
$T_{11}$  & $T_{6}$ & 
\end{tabular}
\caption{The transition function of an FA to accept 
the same language.}
\label{Fig:two}
\end{center}
\end{fig}

You can define different types of new floats, and you can have tables 
of them in the contents pages.


\section{URL's}
Use \verb=\url= from the \textsf{url} package to typeset URL's. Just 
using \verb+\texttt+ or \verb+\tt+ does not work:

\begin{itemize}
\item \verb+\texttt{http://www.mcs.vuw.ac.nz/~neil/}+
\item \verb+\url{http://www.mcs.vuw.ac.nz/~neil/}+
\end{itemize}

Give:
\begin{itemize}
\item \texttt{http://www.mcs.vuw.ac.nz/~neil/}
\item \url{http://www.mcs.vuw.ac.nz/~neil/}
\end{itemize}
If you use the \textsf{hyperref} package then you can produce PDF 
files with clickable hyperlinks using \verb=\url=.

\section{Graphics and \LaTeX}
\LaTeX\ offers rather poor support for the inclusion of graphics. 
There are lots of ways to include pictorial material in \LaTeX, all 
of which are deficient in some way or other. Look at \cite{GRM97GC} for a 
description of them. If your document does need to have pictures in it 
it is worth thinking about what is needed \emph{before} you generate 
the pictures.

\section{The bibliography}

You should build up your bibliography as you go along.  Trying to get 
the details of the bibliography correct at the end of the project is 
hard work. Make sure that you record all the relevant details. Beware 
that material on the internet is likely to change very rapidly. If you 
are going to include material which is only available on the internet, 
then you should probably include in the reference the date on which 
you obtained the document.

\section{Run \LaTeX, run}

\LaTeX\ builds up information about your document for the table of 
contents, references and so on at each run. This means that, for 
example, the table 
of contents is really the table of contents of the previous 
compilation. You may need to run \LaTeX\ two or three times to let it 
catch up with itself. If you have cross references within your 
bibliography (for example two papers from the same collection, such 
as \cite{Dum93a,Dum93b}) you may need to run 
BibTeX more than once. 

It is also possible that the table of contents file has garbage in 
it, and will prevent the document from being compiled. This may 
happen if you have had to abort compilation, due to a bug in the 
source file. If this is the case then removing the \texttt{.toc} file 
will usually solve the problem. You will have to fix the original 
bug, of course.


\section{Find out more by\ldots}
You can find out more by:
\begin{itemize}
\item reading any one of a number of books, such as \cite{GMS94,Lam94}. The 
VUW library has copies of these;
\item visiting  the Comprehensive \TeX\ Archive Network (CTAN) at 
\url{www.ctan.org};
\item typing \texttt{latex} into Google.
\end{itemize}

It is \emph{highly unlikely} that you are the first person who ever 
wanted to do what you want to do with \LaTeX. Therefore it is likely 
that someone has already solved your problem: the real key to using  
\LaTeX\ well is to make effective use of what other people have done.

\section{Summary}
In this chapter we explained some things about \LaTeX.
\chapter{Conclusions}\label{C:con}
The conclusions are presented in this Chapter.



%%%%%%%%%%%%%%%%%%%%%%%%%%%%%%%%%%%%%%%%%%%%%%%%%%%%%%%

\backmatter

%%%%%%%%%%%%%%%%%%%%%%%%%%%%%%%%%%%%%%%%%%%%%%%%%%%%%%%


%\bibliographystyle{ieeetr}
\bibliographystyle{acm}
\bibliography{sample}


\end{document}
