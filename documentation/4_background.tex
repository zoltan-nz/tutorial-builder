\chapter{Background and Related Work}\label{C:background}

During the development and research work, I found a few related interesting projects. These are similar or partly I can use them.

\section{CodeMirror Movie}

This project is interesting, because I found it, when I checked the most popular web-based code editor tool, CodeMirror website. The creator of the CodeMirror wanted to present his tool with a realistic way, so he created CodeMirror Movie. \cite{cm-movie}

This solution highly coupled with CodeMirror, it is like an add-on, so it is possible to attach any CodeMirror implementation. (More about CodeMirror in the next section.)

Adding CodeMirror Movie to our project is quite straightforward. The open source repository provides a CSS and a JS file, which has to add into our page.

This tool is mainly target web-developers, who can add code and scripts to their websites.

Editing "the movie" script is manual. There is a simple syntax which control the presentation steps and this script should be added to the textarea which will be in the code editor.

We clearly see, that it is a very effective way to build a presentation, however it requires real development skills.

Pros:
\begin{itemize}
\item simple, lightweight implementation
\item easy to add your project if you use CodeMirror and you are a developer
\item simple script language to manage the presentation
\item user can use the code editor to try out the presented solution
\end{itemize}
Cons:
\begin{itemize}
\item mainly for developers only
\item highly coupled with CodeMirror
\end{itemize}


\section{Comparison of online code editors}

There are 3 populars web based code editors: CodeMirror, Ace Editor and Monaco.

CodeMirror and Ace Editor are commonly used on websites and different projects. Monaco is a new solution from Microsoft and it is extracted from their popular Microsoft Visual Studio Code developer tool.

There is not significant differences between them. All has the most important code editor features, like supporting more than 100 languages, autocompletion, syntax highlighting, controlling with shortcuts.

I choose CodeMirror, because it has already Ember.js support. Thanks for the ivy-codemirror Ember addon, it can be added to any Ember.js project with the installation of the addon.

\begin{tabular}{| l | l |}
\hline
& CodeMirror \\
\hline
Github link & \url{https://github.com/codemirror/CodeMirror} \\
\hline
Website & \url{http://codemirror.net/} \\
\hline
Popularity (GitHub Star) & 9396 \\
\hline
Ember.js Addon & \url{https://www.emberobserver.com/addons/ivy-codemirror} \\
\hline
\end{tabular}

\begin{tabular}{| l | l |}
\hline
& Ace Editor \\
\hline
Github link & \url{https://github.com/ajaxorg/ace} \\
\hline
Website & \url{https://ace.c9.io} \\
\hline
Popularity (GitHub Star) & 12950 \\
\hline
Ember.js Addon & doesn't exists \\
\hline
\end{tabular}

\begin{tabular}{| l | l |}
\hline
& Monaco \\
\hline
Github link & \url{https://github.com/Microsoft/monaco-editor} \\
\hline
Website & \url{https://microsoft.github.io/monaco-editor/} \\
\hline
Popularity (GitHub Star) & 2322 \\
\hline
Ember.js Addon & doesn't exists \\
\hline
\end{tabular}



\section{Reviewing of code sharing websites}

We can use code editor and sharing platforms also when we want to demo a small feature or describe a problem. These websites are a combination of code editors and an iframe where we can see the preview of the code snippets.

Common feature to split the screen and providing different windows for editing html, css and javascript separately.

User can save the edited content also. Most of them can be embed in a blog post or in other website.

Most important findings:
\begin{itemize}
\item All use Code Mirror as code editor
\item All of them separate the css, html and javascript editing in different screens, but they merge to one file and preview this merged html file in an iframe.
\item Saving the different type of code (css, javascript, html) separately.
\end{itemize}


\begin{tabular}{|l|l|l|l|l|}
\hline
& Code Pen & JSBin & JSFiddle & Ember Twiddle \\
\hline
Link to open source project & not open source & [1] & not open source & [2]  \\
\hline
Website & [3] & [4] & [5] & [6] \\
\hline
Code Editor & Code Mirror & Code Mirror & Code Mirror & Code Mirror \\
\hline
Support embedding & yes & yes & yes & yes \\
\hline
\end{tabular}\\

\\
1 \url{https://github.com/jsbin/jsbin} \\
2 \url{https://github.com/ember-cli/ember-twiddle} \\
3 \url{http://codepen.io} \\
4 \url{http://jsbin.com} \\
5 \url{https://jsfiddle.net/} \\
6 \url{https://ember-twiddle.com/} \\
