\chapter{The problem and the solution}

\section{The problem}

When a developer, teacher or hobbyist would like to present a computer programming problem or solution, most of the times they record a video and publish it on YouTube. Sometimes the easiest way to record a video of a conference/meetup presentation or a live coding session. However recording and editing a high-quality video is time-consuming and less flexible. It is hard to update.

Other problem with showing tutorials with a simple video, that the audience cannot try it out, they have to configure their computer and environment to play with the presented solution.

Most of the cases we would like to show instructions and code snippets, mainly text-based contents. It is preferred to show code snippets in a more realistic environment, in a code editor. So using the online code editor, with a pre-scripted way to play our presentation, where the user can step back and forward and can modify or play with the code is more interactive, involving and helps to understand a problem more clearly.

Additionally, it is much easier to maintain, upgrade or fix for the content creator.

Furthermore, the user can try out and can see the result also, so immediately can practice the new information.

\section{Personas and their goals}

\subsection{Primary persona}

Content creator, open source project maintainer, teacher.

Their motivation is to present a technical problem and the solution clearly and in an easy-to-understand way. The best option showing a demo, what happens if we insert the suggested code, how easy to use it. Most of the times a presentation involves more steps. For example we would like to show a starting state, maybe a few lines of code which we are able to simplify, so in this case the first step is showing the problem and after we show step by step how we solve that.

\subsection{Secondary persona}

The consumer, who watch the presentation, who reads the tutorial and who wants to learn more about the actual problem.

They would like to play, stop, control the presentation. Control the flow, going forward or stepping back.

They would like to try out the solution, for example how the final state changes if they modify the code.

\section{The Solution}

I develop a prototype web app, where the content producer can create a simple step by step tutorial, and the content consumer can "watch" this tutorial and can interact with it.

There are two different users:
\begin{itemize}
\item content producer, let's call teacher
\item content consumer, let's call student
\end{itemize}

\section{Requirements}

Requirements from the teacher perspective:
\begin{itemize}
\item Teacher can navigate to Admin page.
\item Teacher can create a new tutorial.
\item Teacher can add steps to the tutorial.
\item A step could be four different type:
\begin{itemize}
\item Instruction type is a text content.
\item Html type, which adding content to the html editor box.
\item Css type, which adding content to the css editor box.
\item JavaScript type, which adding content to the javascript editor box.
\item Teacher can modify the sort of the steps with reorganizing with drag and drop.
\end{itemize}
\end{itemize}

Requirements from the student perspective:
\begin{itemize}
\item Student can see a list of tutorials.
\item Student can click on a tutorial and can see the steps.
\item Steps are played in sort.
\item Student can "play" and "watch" the steps.
\item Student can "pause" and step "backward".
\end{itemize}

User interface requirements:
\begin{itemize}
\item The tutorial screen has five area:
\item Instruction area
\item Html code editor textarea.
\item Css code editor textarea.
\item Javascript code editor textarea.
\item Html preview textarea.
\item The main website has two main section:
\item Admin page where Teacher can edit tutorials.
\item Tutorials page where Students can select and watch tutorials.
\end{itemize}

